%-*- coding: iso-8859-1 -*-
\documentclass[10pt]{report}

\usepackage[latin1]{inputenc}
\usepackage[T1]{fontenc}
\usepackage{pslatex}

\usepackage[toc=true,style=altlist,number=none]{glossary}
\makeglossary

\title{Title}
\author{Author}
\date{\today}

\makeglossary


\begin{document}

\maketitle

%-*- coding: iso-8859-1 -*-

\storeglosentry{glos:latex}{
name={\LaTeX},
description={The nice \LaTeX{} text processing system},
sort=La}

\storeglosentry{glos:bibtex}{
name={Bibtex},
description={it processes bibliographies, dude "! Note you can add a
  bibliographic entry like this \cite{test06} in glossaries},
sort=Bi}

\storeglosentry{glos:glossary}{
name={Glossary},
description={Yup, we speak about glossaries in the glossary. Pretty
  meta, huh ?}
}



%%% Local Variables: 
%%% mode: latex
%%% TeX-master: "main"
%%% End: 


\chapter{Chapter 1}
\label{sec:chapter-1}

Here we will speak about \gls{glos:latex}.


\chapter{Chapter 2}
\label{sec:chapter-2}

Here we speak about \gls{glos:bibtex}.

\chapter{Chapter 3}
\label{sec:chapter-3}

And here we speak about a \gls{glos:glossary}.


\renewcommand{\glossaryname}{Glossary about stuff}
\printglossary

\bibliography{../../common/bibliography/biblio2}
\bibliographystyle{apalike}

\end{document}




%%% Local Variables: 
%%% mode: latex
%%% TeX-master: t
%%% End: 
